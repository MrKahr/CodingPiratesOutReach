\section{Hvordan det er at være med i Coding Pirates}
\begin{frame}
\frametitle{Hvad laver vi? (i GameDev)}
    \begin{itemize}
        \item Spiludvikling i Unity, GameMaker eller MakeyMake 
        \item Skriver spillogik - hvor højt skal man kunne hoppe, skal der være et dobbelthop?
        \item Laver assets - hvordan skal spilfiguren se ud? Hvem skal dens fjender være? Har vi et tema?
        \item Design - Hvordan skal et level se ud? Hvilke værktøjer skal en spiller have adgang til og hvorfor?
        \item Historiefortælling - Hvordan skal vores karakterer være? Hvad fik dem til at være starte eventyret? 
    \end{itemize}
\end{frame}

\begin{frame}
\frametitle{Hvordan kan man være med?}
    \begin{itemize}
        \item Vær mellem 12-17 år gammel 
        \item Tilmelding på vores hjemmeside\cite{Pirates_tilmeld}
        \item 500kr per sæson (cirka 4 måneder)
        \item Ligesom at gå til musik eller sport - en gang om ugen af cirka 2 timer
        \item Du vælger et projekt du gerne vil lave eller du får en lille opgave, du kan være kreativ med!
        \item Frivillige hjælper med at bygge, hvis du sidder fast - men det er dig, der skal bygge!
        \item Forældre kommer en gang imellem og kigger på projekter, men det er et tilbud til jer!
    \end{itemize}
\end{frame}

\begin{frame}
\frametitle{Demo}
Et produkt fra gamedevafdelingen, som man kan finde på steam!
\begin{figure}
    \href{https://store.steampowered.com/app/1589450/Frogiee/}{\includegraphics[width=\textwidth, keepaspectratio]{/home/madshebsgaard/Projects/CodingPiratesOutReach/assets/pictures/Frogiee_1920x1080.jpg}}
    \caption{Platformerspillet Frogiee af Peter Kragh og Thomas Lauridsen (2021)\cite{Pirates_frogiee}}
    \label{fig:frogiee_screenshot}
\end{figure}
\end{frame}
